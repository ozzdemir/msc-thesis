\documentclass[chaparabic,ee,ms,12pt,oneandhalf,fivejury]{METUThesisTemplate/metu}
\usepackage{appendix}
\usepackage{longtable}
\usepackage[pdftex]{hyperref}
\usepackage[all]{hypcap}
\usepackage{todonotes}
\usepackage{graphicx}
\graphicspath{ {./figures/} }
\usepackage[figuresright]{rotating}
\usepackage{xy} 
\usepackage{booktabs}
\usepackage{pifont}
\usepackage{color}
\usepackage{listings}
\usepackage{pdfpages}
\usepackage{array}
\usepackage{algorithm}
\usepackage{algorithmic}
\usepackage{float}
\usepackage{caption}
\usepackage{lastpage}
\usepackage{afterpage}
\usepackage{lipsum}
\usepackage{adjustbox}
\usepackage{rotating}

% \usepackage{graphicx}
\usepackage{amsmath,amssymb} % define this before the line numbering.
% \usepackage{ruler}
\usepackage{color}
\usepackage{cite}
% \usepackage[utf8x]{inputenc}
% \usepackage{footnote}
% \makesavenoteenv{tabular}
% \makesavenoteenv{table}

\renewcommand{\sectionautorefname}{\S}
\renewcommand{\subsectionautorefname}{\S}

\newcommand{\norm}[1]{\left\lVert#1\right\rVert}

\captionsetup{belowskip=12pt,aboveskip=8pt}
\newcommand{\tab}{\hspace*{2em}}
\DeclareGraphicsExtensions{.pdf,.png,.jpg}


\usepackage{amsmath}
\usepackage{siunitx}
\usepackage{textcomp}
\usepackage{subcaption}


\usepackage{tikz}
\usepackage{mathtools}
% \usepackage{rotating}
%\PassOptionsToPackage{figuresright}{rotating}

%%%%%% Manual package additions start %%%%%%
\usepackage{cleveref}
\usepackage{tikz}
\usepackage{tkz-euclide}
\usetikzlibrary{calc,angles, quotes, intersections}
\pdfstringdefDisableCommands{\let\uppercase\relax} % to silence "Token not allowed in a PDF string" warning
%%%%%% Manual package additions end %%%%%%

\DeclarePairedDelimiter\ceil{\lceil}{\rceil}
\DeclarePairedDelimiter\floor{\lfloor}{\rfloor}


\newcommand{\EA}[1]{\textcolor{red}{[EA: #1]}}

% Name and Surname
\author{Oğuz Özdemir}
% Thesis Title English and Turkish
\title{Feedback Motion Planning of a Novel Fully Actuated Unmanned Surface Vehicle via Sequential Composition of Random Elliptical Funnels}
\turkishtitle{Tam Tahrikli İnsansız Yüzey Aracının Rassal Eliptik Hunilerin Sıralı Bileşimi ile Geri Beslemeli Hareket Planlaması}

\date{Sep 2022}
 
% prof : Prof. Dr.
% assocprof : Assoc. Prof. Dr.
% assistprof : Assist. Prof. Dr.
% dr : Dr.
%
% Director of Institute
\director[prof]{Halil Kalıpçılar}
% Head of Department
\headofdept[prof]{İlkay Ulusoy}
%
% Supervisor : English and Turkish
\supervisor[assistprof]{Mustafa Mert Ankaralı}
% \turkishsupervisor{  } %if you will hard-code the academic title
%
% Affiliation of Supervisor in English and possibly in Turkish
\departmentofsupervisor{Electrical and Electronics Engineering, METU}

%\cosupervisor[dr]{Itır Önal Ertuğrul}
%\departmentofcosupervisor{Robotics Institute, Carnegie Mellon University}
%
% Committee Members
% In general members are sorted according to their academic titles
%
% Proffesors (1)
% Associate Professors (2)
% Assistant Professors (3)
% Other (4)
% 
% IMPORTANT:  All affiliatons should fit in a single line
% If affiliation line is broken into two lines you should shorten the affiliation by using 
% abbrevations or any other means
%
% First committee member should be the chair of examining committee
% Typically the chair is one of the highest ranked committee members
% Ask your supervisor if you are not sure
\committeememberi[assistprof]{Name Surname}
\affiliationi{Computer Engineering, METU}
% Second committee member is always your supervisor
\committeememberii[assistprof]{Mustafa Mert Ankaralı}
\affiliationii{Electrical and Electronics Engineering, METU}
% TODO: Fill also other committee members
% IMPORTANT: If you are Ph.D. student your co-supervisor can not be in your 
% examination committee.

% \def\@proftitlename{Prof. Dr.}\def\@tproftitlename{Prof. Dr.}
% \def\@assocproftitlename{Assoc. Prof. Dr.}\def\@tassocproftitlename{Doç. Dr.}
% \def\@assistproftitlename{Assist. Prof. Dr.}\def\@tassistproftitlename{Yrd. Doç. Dr.}
% \def\@drtitlename{Dr.}\def\@tdrtitlename{Dr.}

\committeememberiii[assistprof]{Name Surname}
\affiliationiii{Computer Engineering, Bilkent University}
% Fourth committee member
\committeememberiv[assistprof]{Gülşah Tümüklü Özyer}
\affiliationiv{Computer Engineering, Atatürk University}
% Fifth committee member
\committeememberv[assistprof]{Jüri}
\affiliationv{JüriBölüm, Ankara University}
%
% Keywords : English & Turkish, Comma seperated
\keywords{Motion and Path Planning, Motion Control, Sequential Composition, Unmanned Surface Vehicles}
\anahtarklm{Hareket ve Yol Planlama, Hareket Kontrolü, Sıralı Ardışık Bileşim, İnsansız Suüstü Araçları}
%
% Abstract in English
%
% Guide on abstract:
% * The structure of the abstract should mirror the structure of the whole thesis, and should represent all its major elements.
% * It should express your thesis (or central idea) and your key points; it should also suggest any implications or applications of the research you discuss in the paper.
\abstract{
    This thesis proposes and analyzes a motion planning and control schema for unmanned surface vehicles that fuses sampling-based approaches' probabilistic completeness with closed-loop approaches' robustness.
    The Proposed schema is based on the sequential composition of elliptical funnels, and it consists of two stages: tree generation and motion control.
    For validation of the approach, we carried out experiments using both simulation and physical setup besides the mathematical analysis. 
    In order to have a common interface for both the simulations and the physical setup and to reduce duplication of work done, we implemented the approach as a ROS (Robot Operating System) node that can interface both similarly.
    Our results show that the proposed method handles the disturbances with minimal disruptions in the stability of the system. Furthermore, elliptic funnels improve the sparsity of the tree compared to the circular ones, thus, resulting in fewer mode changes.
}
%
% Turkish Abstract
%
\oz{
    Bu tez, insansız su üstü araçları için, örneklemeye dayalı yaklaşımların olasılıksal bütünlünü kapalı döngü yaklaşımların gürbüzlüğü ile birleştiren bir hareket planlama ve kontrol şeması önermekte, ve analiz etmektedir.
    Eliptik hunilerin sıralı bileşimine dayandırdığımız çözüm önerisi, ağaç oluşturma ve hareket kontrolü olmak üzere iki aşamadan oluşmaktadır.
    Çalışmamızda, çözümümüzün geçerliliğinin gösterimi için matematiksel analizin yanı sıra hem simülasyon hem de fiziksel sistem kullanılarak deneyler gerçekleştirilmiştir.
    Önerlilen yaklaşım, simülasyon ve fiziksel sistem testlerinde ortak arayüz kullanabilmek ve gereken iş tekrarını azaltmak için, her ikisini de benzer şekilde arayüzleyebilen bir ROS (Robot İşletim Sistemi) düğümü olarak uygulanmıştır.
    Sonuçlarımız, önerilen yöntemin sistem kararlılığında minimum bozulma ile bozucu etkenleri ele aldığını göstermektedir. Ayrıca, eliptik hunilerin, dairesel olanlara kıyasla ağacın seyrekliğini iyileştirerek, hareket kontrolü aşamasında daha az mod değişikliğiyle sonuçlandığı gözlenmiştir.
} 
%
% Dedication 
\dedication{Lorem ipsum dolor sit amet}
%
%
% Acknowledgements   
\acknowledgments{
    \lipsum[1-3]
}

%
% End of Personal and Introductory Information
%%%%%%%%%%%%%%%%%%%%%%%%%%%%%%%%%5
\begin{document}
% Preliminaries
\begin{preliminaries}
    % If you are willing to use any custom stuff before Chapters, put it here
    % Such as List of Abbreviations
    % Check the abbreviations.tex for a template list of abbreviations

    \begin{theglossary}{LONGESTABBRV}

\item[2D] 2 Dimensional
\item[3D] 3 Dimensional 

\end{theglossary}

    % End of Preliminaries
\end{preliminaries}
%   
% Latex content Goes Here 
% 
%

\setlength{\parindent}{0em}
\setlength{\parskip}{10pt}

% You can add as many chapters
\chapter{Introduction}
\label{chapter:introduction}

Motion control and path planning are fundamental problems in robotics, where each tries to solve the problem: how to drive the robot to a target location without colliding with obstacles.
In the literature, different approaches exist that propose solutions to these problems with different assumptions, limitations, and resource requirements concerning environmental representation and motion models.

The open-loop approach to this problem is to create an open-loop trajectory to be tracked by a closed-loop control strategy. Sampling-based planners such as PRM \cite{Kavraki1996}, RRT \cite{rrt2001}, and EST \cite{Est1999} are well-known examples of open-loop planners.
Later in the trajectory tracking stage, generated online path trackers follow a collision-free and possibly smooth path \cite{camacho2004} \cite{Amidi1991}.
Generally speaking, sampling-based planners' trajectories are not required to be smooth and may require an additional smoothing layer before the path following stage. % TODO: May be extended with steering / forward propagation functions, i.e. control-based planning, usage with sampling based planners which results in smoother and more viable trajectories. Ref: Section 7.5.1 of \cite{Lipson2001}

The closed-loop approach to the path planning and motion control problem requires the creation of a control policy, which would bring the agent to the goal configuration from any valid initial configuration.
This approach ensures that the agent would reach the goal configuration even if it diverges from the optimal trajectory due to the disturbances.
The most common realizations of the closed-loop approach are based on potential functions, with control policies based on gradient descent \cite{Khatib1986}. % Refs here, att-rep, or any other potential field planners, that is prone to local-minimas.
There also exist planners based on navigation functions, which are potential functions with single stable local minima at the goal configuration \cite{KODITSCHEK1990412}.
Navigation function methods are immune to the local-minima problem. However, their application is limited to simple environment representations such as sphere worlds and star worlds \cite{Rimon1992}, with limited real-world use.
The use of harmonic functions for global navigation is another local-minima-free alternative \cite{Connolly1990} \cite{Golan2017}, but the required processing power for numerical solution in these approaches is enormous and not suitable for real-time onboard computing.
Since creating a single lightweight control policy over the whole configuration space that would forward the agent to the goal configuration while avoiding collisions is a challenging (and mostly impossible) task, hybrid approaches also exist.

Hybrid methods utilize the open-loop approach's simplicity with the closed-loop approach's robustness.
Sampling-based neighborhood graph constructs a graph whose deployments probabilistically cover the configuration space in its initialization phase \cite{sng2004}.
The constructed graph is later used to find the paths between the goal deployment and any other deployment, and local navigation functions are applied to each deployment.
This approach enables the use of compositions of navigation functions in rather complex environments. Work presented in \cite{karagoz2020MpcGraph} follows a similar schema, but it uses Model Predictive Control (MPC) for the motion control inside the deployments.
Schemas presented in \cite{golbol2018RGTrees} and \cite{Ege2019}, on the other hand, uses the ideas given in \cite{Burridge1999SequentialComposition} and \cite{rrt2001} to create a funnel tree with a lightweight control policy defined over the funnels.
Another example of a hybrid method that creates a collision-free feedback law over given cell decompositions is proposed in \cite{Lindemann2009}.

\section{Contributions}
\label{section:contributions}

Our contributions are as follows:

\begin{itemize}
    \item There is no one who loves pain itself, who seeks after it and wants to have it, simply because it is pain
    \item There is no one who loves pain itself, who seeks after it and wants to have it, simply because it is pain
    \item There is no one who loves pain itself, who seeks after it and wants to have it, simply because it is pain
\end{itemize}

\section{The Outline of the Thesis}
\label{section:outline}
\lipsum[1-3]

\chapter{Methodology} % TODO: Can be renamed later on
\label{chapter:methodology}

\section{Tree Generation}
\label{section:tree-generation}
\lipsum

\section{Motion Control}
\label{section:motion-control}
\lipsum

\chapter{Simulation Results}
\label{chapter:simulation-results}

\chapter{Experimental Setup}
\label{chapter:experimental-setup}

\chapter{Experimental Results}
\label{chapter:experimental-results}

\chapter{Conclusion and Future work}
\label{chapter:conclusion-and-futurework}

\section{Conclusion}
\label{section:conclusion}
\lipsum[1-3]

\section{Future Work}
\label{section:futurework}
\lipsum[1-3]

\bibliographystyle{ieeetr} 
\bibliography{thesis} 

%
% References in Bibtex format goes into below indicated file with .bib extension
%\bibliography{thesis_references}
% You can use full name of authors, however most likely some of the Bibtex entries you will find, will use abbreviated first names
% If you don't want to correct each of them by hand, you can use abbreviated style for all of the references

%\bibliographystyle{abbrv}

% if you have more that one appendix, then use \appendices, otherwise use 
% \appendix
% \input{appendix1/appendix1.tex}
\end{document}
